\documentclass[10pt]{beamer}

\usetheme[progressbar=frametitle]{metropolis}
\usepackage{appendixnumberbeamer}

\usepackage{booktabs}
\usepackage[scale=2]{ccicons}

\usepackage{pgfplots}
\usepgfplotslibrary{dateplot}

\usepackage{xspace}
\newcommand{\themename}{\textbf{\textsc{metropolis}}\xspace}

\usepackage{graphicx}
\graphicspath{ {pics/} }

\usepackage{listings}
\definecolor{mGreen}{rgb}{0,0.6,0}
\definecolor{mGray}{rgb}{0.5,0.5,0.5}
\definecolor{mPurple}{rgb}{0.58,0,0.82}
\definecolor{backgroundColour}{rgb}{0.95,0.95,0.92}

\lstdefinestyle{CStyle}{
    backgroundcolor=\color{backgroundColour},   
    commentstyle=\color{mGreen},
    keywordstyle=\color{magenta},
    numberstyle=\tiny\color{mGray},
    stringstyle=\color{mPurple},
    basicstyle=\footnotesize,
    breakatwhitespace=false,         
    %breaklines=true,                 
    captionpos=b,                    
    keepspaces=true,                 
    numbers=left,                    
    numbersep=5pt,                  
    showspaces=false,                
    showstringspaces=false,
    showtabs=false,                  
    tabsize=2,
    language=C
}

%----------------------------------------------------------------------------------------
%	TITLE PAGE
%----------------------------------------------------------------------------------------

\title{Software Security}
\subtitle{Kryptographie und IT Sicherheit SS 2018}
% \date{\today}
\date{}
\author{Dimitrii ,Manuel Klappacher}
\institute{Universit\"at Salzburg}
% \titlegraphic{\hfill\includegraphics[height=1.5cm]{logo.pdf}}

\begin{document}

\maketitle

\begin{frame}{Themen}
  \setbeamertemplate{section in toc}[sections numbered]
  \tableofcontents[hideallsubsections]
\end{frame}

%----------------------------------------------------------------------------------------
%	Einleitung
%----------------------------------------------------------------------------------------

\section{Einleitung}

\begin{frame}[fragile]{Einleitung}
  Wie entstehen Sicherheitsl\"cken?
  \begin{itemize}
    \item Programmierfehler
      \begin{itemize}
        \item Treten sehr h\"aufig auf
        \item Logische Fehler, syntaktische Fehler, lexikalische Fehler
        \item Zeitdruck
        \item Mangelnde Kenntniss
        \item Keine ausreichenden Tests
      \end{itemize}
    \item Compilerfehler
      \begin{itemize}
        \item Treten nicht sehr h\"aufig auf
      \end{itemize}
    \item Absichtlich platzierte Backdoors
      \begin{itemize}
        \item Sehr schwer nachzuweisen - wie Unterscheidet man Fehler von b\"oswilliger Absicht?
        \item Werden acuh von anderen Teilnehmern entdeckt und von Kriminellen dann f\"ur ihre Zwecke missbraucht
      \end{itemize}
  \end{itemize}
\end{frame}

%----------------------------------------------------------------------------------------
%	REMOTE AND LOCAL THREATS
%----------------------------------------------------------------------------------------

\section{Remote and Lokale Gefahren}

%----------------------------------------------------------------------------------------
%	EXPLOITS
%----------------------------------------------------------------------------------------

\section{Exploits}

\begin{frame}[fragile]{Code Injection}
  Code Injection ist das ausnutzen von Bugs durch Eingabe von ungewollten Parametern, um dadurch die Ausf\"urung zu ver\"andern.
  \newline
  Kann folgende Auswirkungen haben:
  \begin{itemize}
    \item Daten in SQL Tabellen verändern
    \item Installieren von Malware durch Server-Scripting Code zB. PHP
    \item Root Privilegien bekommen, durch Shell Injection oder Windows Service
    \item Angtiff auf Web User durch Cross-Site-Scripting in HTML/JS
  \end{itemize}
\end{frame}

\begin{frame}[fragile]{Code Injection - Masnahmen}
  Kann erschwert werden durch:
  \begin{itemize}
    \item API's benutzen, die sicher gegen\"uber allen Symbolen sind, indem der Eingabestring compiliert und gefiltert wird.
    \item Whitelisting von erw\"unschten Parametern
  \end{itemize}
\end{frame}

\begin{frame}[fragile]{SQL Injection}
  Ausnutzen von Sicherheitsl\"ucken in Zusammenhang mit SQL-Datenbanken.
  Ziele:
  \begin{itemize}
    \item Daten auszusp\"ahen oder zu ver\"andern
    \item Kontrolle \"uber Server zu erhalten
  \end{itemize}
\end{frame}

\begin{frame}[fragile]{SQL Injection - Beispiel}
  Es wird zus\"atzlicher Code bei Aufruf eingeschleust, der die Bentzertabelle modifiziert.
  \newline
  \includegraphics[scale=0.5]{sql_injection_1}
\end{frame}

\begin{frame}[fragile]{Cross Site Scripting}
  test frame
\end{frame}

\begin{frame}[fragile]{Cross Site Scripting - reflektierte Angriffe}
  Eine Benutzereingabe wird direkt vom Server wieder zur\"uck gesendet.
  Wenn diese Eingabe Scriptcode enth\"alt, die vom Browser des Nutzers interpretiert wird, kann dorrt Schadcode ausgef\"urt werden.
  Beispiel: Suchfunktion.
  \lstset{language=html, basicstyle=\tiny}
  \begin{lstlisting}
    http://example.com/?suche=Suchbegriff
    http://example.com/?suche=<script type="text/javascript">alert("XSS")</script>
    <p>Sie suchten nach: <script type="text/javascript">alert("XSS")</script></p>
  \end{lstlisting}
  Ausgenutzt wird das dynamisch generierte Websiten ihren Ihnalt an \"ubergebene Eingabewerte anpassen, durch HTTP-GET und HTTP-POST.
  Dieser Typ heisst auch nicht-persistent, da der Schadcode nur tempr\"ar bei der jeweiligen Generierung der Website eingeschleust wird.
\end{frame}

\begin{frame}[fragile]{Cross Site Scripting - persistente Angriffe}
  Unterscheidet sich von reflektierenden Angriffen nur dadurch, dass der Schadcode auf dem Server gespeichert wird, wodurch er bei jeder Anfrage asugef\"urt wird.
  Ist bei Webanwendungen m\"oglich, die Benutzereingaben serverseitig ohne Pr\"ufung speichern und diese sp\"ater wieder ausliefert.
  Beispiel Posting auf Website:
  \lstset{language=html, basicstyle=\tiny}
  \begin{lstlisting}
    Eine sehr gutes Produkt!<script type="text/javascript">alert("XSS")</script>
  \end{lstlisting}
\end{frame}

\begin{frame}[fragile]{Cross Site Scripting - DOM-basierte Angriffe}
  Webapplikation auf dem Server ist hier nicht beteiligt, wird auch lokales XSS genannt.
  Somit auch statische HTML Seiten mit JavaScript unterst\"utzung anf\"llig f\"ur diesen Angriff.
\end{frame}

\begin{frame}[fragile]{Cross Site Scripting - Schutzma{\ss}nahmen}
  \begin{itemize}
    \item Anstatt Blacklist mit b\"sen Eingaben zu f\"uhren, besser Whitelist mit buten Eingaben. Da die Anzahl der Angriffmethoden nicht bekannt ist.
    \item HTML-Metazeichen durch Zeichenreferenzen ersetzen, damit sie als normale Zeichen behandelt werden
    \item Sicher programmierte Anwendung sind Web Apllication Firewalls (WAF) vorzuziehen.
  \end{itemize}
\end{frame}

\begin{frame}[fragile]{Directory Traversal Attack}
  Ein HTTP Angriff, bei dem ein Angreifer zugriff auf gesperrte Verzeichnisse gewinnt und Code auserhalb des root Verzeichnisses ausf\"uhrt.
\end{frame}

\begin{frame}[fragile]{Directory Traversal Attack - Schutzma{\ss}nahmen}
  test frame
\end{frame}

\begin{frame}[fragile]{Format String}
  test frame
\end{frame}

\begin{frame}[fragile]{Format String - Schutzma{\ss}nahmen}
  test frame
\end{frame}

\begin{frame}[fragile]{Buffer Overflows}
  test frame
\end{frame}

\begin{frame}[fragile]{Buffer Overflows - Type-Safe Sprachen}
  Compiler stellt Typsicherheit her, indem Datentypen gepr\"uft werden, damit keine Typverletzungen entstehen.
  Wenn Typverletzungen sp\"atestens zur Laufzeit erkannt werden, spricht man von Typsicheren Programmiersprachen.
  
  Beispiel String in Python, es reicht der Variable einen String zuzuweisen.
  %\lstset{language=python}
  \begin{lstlisting}[style=CStyle]
    mystring = "This is my string"
  \end{lstlisting}
  
  Beispiel in C, es muss der Typ deklariert und auch der Speicher manuell reserviert werden.
  \begin{lstlisting}[style=CStyle]
    char mystring[20] = "This is my string";
  \end{lstlisting}
  Wenn man in C nun einen 30 Byte String zuweist entsteht eine Overflow Situation.
\end{frame}

\begin{frame}[fragile]{Buffer Overflows - Schutzma{\ss}nahmen}
  \begin{itemize}
    \item Type-Safe Programmiersprachen verwenden, welche Memory Management zB Java, Python, Ruby,...
    \item \"Uberpr\"ufen auf Overflows bei User Eingaben
    \item in C sichere Methoden verwenden, \textit{get\_s} anstatt \textit{get}. 
  \end{itemize}
\end{frame}



%----------------------------------------------------------------------------------------
%	OPEN SOURCE UND PROPERTÄRE SOFTWARE
%----------------------------------------------------------------------------------------

\section{Open Source und Propert\"are Software}

%----------------------------------------------------------------------------------------
%	FIRMWARE SECURITY
%----------------------------------------------------------------------------------------

\section{Firmware Security}
\begin{frame}[fragile]{Spezialfall Firmware}
  \begin{itemize}
    \item Ger\"at wurde bereits verkauft, kein Interesse des Herstellers an Updates
    \item Zu viele verschiedene Ger\"ate - Unm\"oglicher Verwaltungsaufwand
      \begin{itemize}
        \item Alleine Samsung hat bis 2014 56 verscheidene Smartphones pro Jahr herausgebracht
      \end{itemize}
    \item Firmware agiert in Schicht unter Betriebssystem - Angriffe k\"onnen vom Benutzer nicht erkannt oder verhindert werden
    \item Firmware meist Closed Source - keine Weiterentwicklung der Community
  \end{itemize}
\end{frame}

\begin{frame}[fragile]{Sicherheitsl\"ucken in Firmware - Beispiele}
  \begin{itemize}
    \item BadUSB - Eingabeger\"ate, USB-Sticks, Speichermedien, Kameras, ...
    \item Intel ME - Betriebssystem im Prozessor (AMD PSP)
      \begin{itemize}
        \item Funktionsweise undokumentiert
        \item Kritische L\"ucke 2017 entdeckt
        \item NSA und Google haben Intel ME abgeschaltet auf ihren Ger\"aten
      \end{itemize}
    \item Android
      \begin{itemize}
        \item praktisch alle Android Ger\"ate ohne Sicherheitupdates
      \end{itemize}
    \item Router, Smart TV's, IoT-Devices - Millionen angreifbare Ger\"ate in Haushalten, Firmen und Beh\"orden
  \end{itemize}
\end{frame}

\begin{frame}[fragile]{Smartphones Sicherheitsupdates}
  \includegraphics[scale=0.5]{android_sub}
  \newline
\end{frame}


\end{document}
