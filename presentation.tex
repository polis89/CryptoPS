\documentclass[10pt]{beamer}

\usetheme[progressbar=frametitle]{metropolis}
\usepackage{appendixnumberbeamer}

\usepackage{booktabs}
\usepackage[scale=2]{ccicons}

\usepackage{pgfplots}
\usepgfplotslibrary{dateplot}

\usepackage{xspace}
\newcommand{\themename}{\textbf{\textsc{metropolis}}\xspace}

%----------------------------------------------------------------------------------------
%	TITLE PAGE
%----------------------------------------------------------------------------------------

\title{Software Security}
\subtitle{Kryptographie und IT Sicherheit SS 2018}
% \date{\today}
\date{}
\author{Dimitrii ,Manuel Klappacher}
\institute{Universit\"at Salzburg}
% \titlegraphic{\hfill\includegraphics[height=1.5cm]{logo.pdf}}

\begin{document}

\maketitle

\begin{frame}{Themen}
  \setbeamertemplate{section in toc}[sections numbered]
  \tableofcontents[hideallsubsections]
\end{frame}

%----------------------------------------------------------------------------------------
%	Einleitung
%----------------------------------------------------------------------------------------

\section{Einleitung}

\begin{frame}[fragile]{Einleitung}
  test frame
\end{frame}

%----------------------------------------------------------------------------------------
%	REMOTE AND LOCAL THREATS
%----------------------------------------------------------------------------------------

\section{Remote and Lokale Gefahren}

%----------------------------------------------------------------------------------------
%	EXPLOITS
%----------------------------------------------------------------------------------------

\section{Exploits}

\begin{frame}[fragile]{Code Injection}
  Code Injection ist das ausnutzen von Bugs durch Eingabe von ungewollten Parametern, um dadurch die Ausf\"urung zu ver\"andern.
  \newline
  Kann folgende Auswirkungen haben:
  \begin{itemize}
    \item Daten in SQL Tabellen verändern
    \item Installieren von Malware durch Server-Scripting Code zB. PHP
    \item Root Privilegien bekommen, durch Shell Injection oder Windows Service
    \item Angtiff auf Web User durch Cross-Site-Scripting in HTML/JS
  \end{itemize}
\end{frame}

\begin{frame}[fragile]{Code Injection - Masnahmen}
  Kann erschwert werden durch:
  \begin{itemize}
    \item API's benutzen, die sicher gegen\"uber allen Symbolen sind, indem der Eingabestring compiliert und gefiltert wird.
    \item Whitelisting von erw\"unschten Parametern
  \end{itemize}
\end{frame}

\begin{frame}[fragile]{SQL Injection}
  test frame
\end{frame}

\begin{frame}[fragile]{Cross Site Scripting}
  test frame
\end{frame}

\begin{frame}[fragile]{Directory Traversal Attack}
  test frame
\end{frame}

\begin{frame}[fragile]{Format String}
  test frame
\end{frame}

\begin{frame}[fragile]{Buffer Overflows}
  test frame
\end{frame}

%----------------------------------------------------------------------------------------
%	OPEN SOURCE UND PROPERTÄRE SOFTWARE
%----------------------------------------------------------------------------------------

\section{Open Source und Propert\"are Software}

%----------------------------------------------------------------------------------------
%	FIRMWARE SECURITY
%----------------------------------------------------------------------------------------

\section{Firmware Security}

\end{document}
